\documentclass[man]{apa6}
\usepackage{lmodern}
\usepackage{amssymb,amsmath}
\usepackage{ifxetex,ifluatex}
\usepackage{fixltx2e} % provides \textsubscript
\ifnum 0\ifxetex 1\fi\ifluatex 1\fi=0 % if pdftex
  \usepackage[T1]{fontenc}
  \usepackage[utf8]{inputenc}
\else % if luatex or xelatex
  \ifxetex
    \usepackage{mathspec}
  \else
    \usepackage{fontspec}
  \fi
  \defaultfontfeatures{Ligatures=TeX,Scale=MatchLowercase}
\fi
% use upquote if available, for straight quotes in verbatim environments
\IfFileExists{upquote.sty}{\usepackage{upquote}}{}
% use microtype if available
\IfFileExists{microtype.sty}{%
\usepackage{microtype}
\UseMicrotypeSet[protrusion]{basicmath} % disable protrusion for tt fonts
}{}
\usepackage{hyperref}
\hypersetup{unicode=true,
            pdftitle={APA Midterm, Reproducing The Analysis of Ella L. James et al(2015)},
            pdfauthor={Norlander Wilson Shekoni~\&},
            pdfkeywords={intrusive memory, intrusions, reconsolidation, computer game,
involuntary memory, trauma film, mental imagery, emotion, open data,
open materials},
            pdfborder={0 0 0},
            breaklinks=true}
\urlstyle{same}  % don't use monospace font for urls
\usepackage{graphicx,grffile}
\makeatletter
\def\maxwidth{\ifdim\Gin@nat@width>\linewidth\linewidth\else\Gin@nat@width\fi}
\def\maxheight{\ifdim\Gin@nat@height>\textheight\textheight\else\Gin@nat@height\fi}
\makeatother
% Scale images if necessary, so that they will not overflow the page
% margins by default, and it is still possible to overwrite the defaults
% using explicit options in \includegraphics[width, height, ...]{}
\setkeys{Gin}{width=\maxwidth,height=\maxheight,keepaspectratio}
\IfFileExists{parskip.sty}{%
\usepackage{parskip}
}{% else
\setlength{\parindent}{0pt}
\setlength{\parskip}{6pt plus 2pt minus 1pt}
}
\setlength{\emergencystretch}{3em}  % prevent overfull lines
\providecommand{\tightlist}{%
  \setlength{\itemsep}{0pt}\setlength{\parskip}{0pt}}
\setcounter{secnumdepth}{0}
% Redefines (sub)paragraphs to behave more like sections
\ifx\paragraph\undefined\else
\let\oldparagraph\paragraph
\renewcommand{\paragraph}[1]{\oldparagraph{#1}\mbox{}}
\fi
\ifx\subparagraph\undefined\else
\let\oldsubparagraph\subparagraph
\renewcommand{\subparagraph}[1]{\oldsubparagraph{#1}\mbox{}}
\fi

%%% Use protect on footnotes to avoid problems with footnotes in titles
\let\rmarkdownfootnote\footnote%
\def\footnote{\protect\rmarkdownfootnote}


  \title{APA Midterm, Reproducing The Analysis of Ella L. James et al(2015)}
    \author{Norlander Wilson Shekoni\textsuperscript{}~\&}
    \date{}
  
\shorttitle{Game Play Reduces Intrusive Memories}
\affiliation{
\vspace{0.5cm}
\textsuperscript{1} Brooklyn College City University Of New York\\\textsuperscript{} }
\keywords{intrusive memory, intrusions, reconsolidation, computer game, involuntary memory, trauma film, mental imagery, emotion, open data, open materials\newline\indent Word count: X}
\usepackage{csquotes}
\usepackage{upgreek}
\captionsetup{font=singlespacing,justification=justified}

\usepackage{longtable}
\usepackage{lscape}
\usepackage{multirow}
\usepackage{tabularx}
\usepackage[flushleft]{threeparttable}
\usepackage{threeparttablex}

\newenvironment{lltable}{\begin{landscape}\begin{center}\begin{ThreePartTable}}{\end{ThreePartTable}\end{center}\end{landscape}}

\makeatletter
\newcommand\LastLTentrywidth{1em}
\newlength\longtablewidth
\setlength{\longtablewidth}{1in}
\newcommand{\getlongtablewidth}{\begingroup \ifcsname LT@\roman{LT@tables}\endcsname \global\longtablewidth=0pt \renewcommand{\LT@entry}[2]{\global\advance\longtablewidth by ##2\relax\gdef\LastLTentrywidth{##2}}\@nameuse{LT@\roman{LT@tables}} \fi \endgroup}


\DeclareDelayedFloatFlavor{ThreePartTable}{table}
\DeclareDelayedFloatFlavor{lltable}{table}
\DeclareDelayedFloatFlavor*{longtable}{table}
\makeatletter
\renewcommand{\efloat@iwrite}[1]{\immediate\expandafter\protected@write\csname efloat@post#1\endcsname{}}
\makeatother
\usepackage{lineno}

\linenumbers

\authornote{A Brooklyn College Graduate Student. Also, a
Brooklyn College Research Assistant.

Correspondence concerning this article should be addressed to Norlander
Wilson Shekoni, 2900 Bedford Ave, Brooklyn NY, 11210. E-mail:
\href{mailto:norlander3@gmail.com}{\nolinkurl{norlander3@gmail.com}}}

\abstract{
A reproduction of the analysis for Experiment 1 from Ella L. James
Michael B. Bonsall, Laura Hoppitt, Elizabeth M. Tunbridge, John R.
Geddes, Amy L. Milton.

This report re-produces the analysis of Experiment 1 reported in Ella
L.James and John R. Geddes (2015). The data was downloaded from
(``\url{https://raw.githubusercontent.com/CrumpLab/statisticsLab/master/data/Jamesetal2015Experiment2.csv}'')

Analyzed two trial gatherings, anticipating that a gathering that
finished a memory-reactivation task in addition to Tetris amusement play
would demonstrate a lower recurrence of meddling recollections of an
awful film, compared with a control bunch given no undertakings. The
reactivation-in addition to Tetris gathering (n = 26) finished a
memory-reactivation task---introduction of 11 film stills pursued by a
filler task for 10 min and afterward played Tetris for 12 min. The
control gathering (n = 26) was neither given the memory-reactivation
task nor played Tetris; rather, after the 10-min filler task, they had a
12-min break in which there was no undertaking. Along these lines, the
two gatherings kept on chronicle meddling recollections for 7 (Days 1--
7). they predicted that reconsolidation of a reactivated visual memory
of experimental trauma could be disrupted by engaging in a visuospatial
task that would compete for visual working memory resources.


}

\begin{document}
\maketitle

\section{Methods}\label{methods}

\subsection{Participants}\label{participants}

There were 69 partcipants. 26 per section.

\subsection{Material}\label{material}

The details of the Computer Game Play Reduces Intrusive Memories of
Experimental Trauma via Reconsolidation-Update Mechanisms are in the
report of James et al. (2015).

\subsection{Procedure}\label{procedure}

This test included three research facility sessions just as the fruition
of a pen-and-paper journal at home to record the every day recurrence of
meddlesome recollections (both more than 24 hr and after that for an
extra 7 days).

\section{Results}\label{results}

Means for each subject in each conditon in a one factor (Control vs
Reactivation Plus Tetris ) before intervention and ((Tetris only and
Reactivation only) vs Control) after intervention.

Were submitted to a one factor ANOVA. Means results are displayed in
Table 1 and Figure 1. The full ANOVA table is reported in Table 2.

\section{Discussion}\label{discussion}

The re-analysis successfully reproduced the reported James et al.

We presented the mean nosy recollections for the week from each subject
in each condition to a one-factor between subjects ANOVA, with
Intervention type (No-task control, Reactivation Plus tetris, Tetris
just, Reactivation just) as the sole free factor. We found a primary
impact of Intervention type, F(3, 68) = 3.79, MSE = 10.09, p = 0.014.
Mean meddlesome recollections were essentially extraordinary between the
Control (M = 5.11, SE = .99), Reactivation in addition to Tetris (M =
3.89, SE = .68), Tetris just (M= 3.89, SE = .68), and Reactivation just
(M = 4.83, .78) conditions

\newpage

Figure 1 Graph of Conditions VS Means for Experiment 1
\includegraphics{testMidterm_files/figure-latex/unnamed-chunk-2-1.pdf}

\newpage

Table 1

Means of Intervention Before and After Experiment 1

\begin{tabular}{l|r|r}
\hline
Condition & means & SEs\\
\hline
Control & 5.111111 & 0.9963623\\
\hline
Reactivation+Tetris & 1.888889 & 0.4113495\\
\hline
Tetris\_only & 3.888889 & 0.6806806\\
\hline
Reactivation\_only & 4.833333 & 0.7848650\\
\hline
\end{tabular}

\newpage

Table 2 ANOVA Table for Experiment 1.

\begin{tabular}{l|r|r|r|r|r}
\hline
  & Df & Sum Sq & Mean Sq & F value & Pr(>F)\\
\hline
Condition & 3 & 114.8194 & 38.27315 & 3.794762 & 0.0140858\\
\hline
Residuals & 68 & 685.8333 & 10.08578 & NA & NA\\
\hline
\end{tabular}

\newpage 

\section{Power Analysis}\label{power-analysis}

Figure 2 Power Analysis
\includegraphics{testMidterm_files/figure-latex/unnamed-chunk-5-1.pdf}

\newpage

\section{References}\label{references}

\begingroup
\setlength{\parindent}{-0.5in} \setlength{\leftskip}{0.5in}

\hypertarget{refs}{}
\hypertarget{ref-james2015computer}{}
James, E. L., Bonsall, M. B., Hoppitt, L., Tunbridge, E. M., Geddes, J.
R., Milton, A. L., \& Holmes, E. A. (2015). Computer game play reduces
intrusive memories of experimental trauma via reconsolidation-update
mechanisms. \emph{Psychological Science}, \emph{26}(8), 1201--1215.

\endgroup


\end{document}
